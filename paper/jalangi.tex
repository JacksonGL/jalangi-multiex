\documentclass{sig-alternate}

\usepackage{xspace}
\usepackage{algorithmic}
\usepackage{algorithm}
\usepackage{listings}
\usepackage{float}
\usepackage{soul}
\usepackage{multirow}
\usepackage{url}

\floatstyle{boxed}
\restylefloat{figure}
\setlength{\floatsep}{0pt}
\renewcommand{\topfraction}{0.95}
\renewcommand{\textfraction}{0.05}
\renewcommand{\floatpagefraction}{0.95}

\def\jalangi{\textsc{Jalangi}}
\newcommand \dsl [1] {\ensuremath{{\tt #1}}\xspace}
\newcommand \usl [1] {\mbox{\underline{\tt #1}}\xspace}
\newcommand \Sync{\dsl{sync}}
\newcommand \Actual{\usl{a}}
\newcommand \Shadow{\usl{s}}
\newcommand \Enter{\dsl{enter}}
\newcommand \Exit{\dsl{exit}}
\newcommand \analysis{\usl{anlys}}

\lstdefinelanguage{JavaScript}{
  keywords={typeof, new, true, false, catch, function, return, null, catch, switch, var, if, in, while, do, else, case, break},
  keywordstyle=\bf\tt,
  ndkeywords={anlys, literal, binary, unary, putField,
    getField, conditional, call},
  ndkeywordstyle=\underline,
  sensitive=false,
  comment=[l]{//},
  morecomment=[s]{/*}{*/},
  commentstyle=\sl,
  stringstyle=\sf,
  morestring=[b]',
  morestring=[b]"
}

\lstdefinelanguage{pseudo} { 
  	basicstyle=\tt, 
	keywordstyle=\bf\tt,
  	morekeywords= { Object, foreach, to, for, new, while, do, if, then, else, return, caoea, int, in  }, 
    sensitive=false, 
    morecomment=[l]{//}, 
    morecomment=[s]{/*}{*/},
	morestring=[b]"
}

\begin{document}



\conferenceinfo{Under Submission}{Do Not Distribute}

\title{Jalangi: A Selective Record-Replay and Dynamic Analysis
  Framework for JavaScript} \numberofauthors{2}

\author{
\alignauthor
Koushik Sen\titlenote{The work of this author was supported in full by
  Samsung Research America.}\\
      \affaddr{EECS Department}\\
      \affaddr{UC Berkeley, CA, USA.}\\
     \affaddr{{\tt\small ksen@cs.berkeley.edu}}
\alignauthor
Swaroop Kalasapur, Tasneem Brutch, and Simon Gibbs\\
\affaddr{Samsung Research America}\\
\affaddr{1732 North First Street, San Jose, CA, USA}\\
\affaddr{{\tt\small \{s.kalasapur,t.brutch,s.gibbs\}@sisa.samsung.com}}
}
% 2nd. author
\maketitle
\sloppy

\begin{abstract}
  JavaScript is widely used for writing client-side web applications
  and is getting increasingly popular for writing mobile applications.
  However, unlike C, C++, and Java, there are not that many tools
  available for analysis and testing of JavaScript applications.  In
  this paper, we present a simple yet powerful framework, called
  \jalangi{}, for writing heavy-weight dynamic analyses.  Our
  framework incorporates two key techniques: 1) selective
  record-replay, a technique which enables to record and to faithfully
  replay a user-selected part of the program, and 2) shadow values and
  shadow execution, which enables easy implementation of heavy-weight
  dynamic analyses.  Our implementation makes no special assumption
  about JavaScript, which makes it applicable to real-world JavaScript
  programs running on multiple platforms.  We have implemented
  concolic testing, an analysis to track origins of nulls and
  undefined, a simple form of taint analysis in \jalangi{}, an
  analysis to detect likely type inconsistencies, and an object
  allocation profiler.  Our evaluation of \jalangi{} on the SunSpider
  benchmark suite and on five web applications shows that \jalangi{}
  has an average slowdown of 26X during recording and 30X slowdown
  during replay and analysis.  The slowdowns are comparable with
  slowdowns reported for similar tools, such as PIN and Valgrind for
  x86 binaries.  We believe that the techniques proposed in this paper
  are applicable to other dynamic languages.
\end{abstract}

\section{Introduction}

JavaScript is the most popular programming language for client-side
web programming. Advances in browser technologies and JavaScript
engines in the recent years have fueled the use of JavaScript in Rich
Internet Applications, and several mobile platforms including Android,
iOS, Tizen, Windows8, Blackberry, support applications written in
HTML5/JavaScript. A key reason behind the popularity of JavaScript
programs is that the are portable. Once written, JavaScript based
applications can be executed on any platform that has a web browser
with JavaScript support, which is quite common in modern day devices.
JavaScript being a dynamic language, also attracts developers through
its flexible features that do not require explicit memory management,
static typing and compilation.  With a renewed interest in JavaScript,
many complex applications such as Google docs, gmail, and a variety of
games are being developed using HTML5/JavaScript. However, unlike
C/C++, Java and C\#, JavaScript is significantly shorthanded in the
tools landscape. The dynamic and reflective nature of JavaScript makes
it hard to analyze it
statically~\cite{Richards:2010:ADB:1806596.1806598,Wei:2012:BAJ:2384716.2384758,Ratanaworabhan:2010:JCB:1863166.1863169}.

In this paper, we present a dynamic analysis framework, called
\jalangi{}, for Javacript.  The framework provides a few useful
abstractions and an API that significantly simplifies implementation
of dynamic analyses for JavaScript.  The framework works through
source code instrumentation and allows implementation of various
heavy-weight dynamic analyses techniques.  \jalangi{} incorporates two
ideas:

\begin{enumerate}
\item \emph{Selective record-replay}, a technique which enables to record and
  to faithfully replay a user-selected part of the program.  For
  example, if a JavaScript application, uses several third-party
  modules, such as jQuery, Box2DJS, along with an application specific
  library called myapp.js, our framework enables us to only record and
  replay the behavior of myapp.js.
\item\emph{ Shadow values}, which enables us to associate a shadow
  value with any value used in the program.  A shadow value can
  contain useful information about the actual value (e.g. taint
  information or symbolic representation of the actual value).  The
  framework supports \emph{shadow execution} on shadow values, a
  technique in which an analysis can update the shadow values and
  analysis state, on each operation performed by the actual execution.
  For example, a shadow execution can perform symbolic execution or
  dynamic taint propagation.
\end{enumerate}

There are a few constraints which dictated the design of the above
techniques in \jalangi{}.

\begin{enumerate}
\item We wanted to design a framework that is independent of browsers
  and JavaScript engines.  Such a design enables us to design dynamic
  analyses that are not tied to a particular JavaScript engine.
  Independence from browsers and JavaScript engines also enables us to
  easily maintain our framework in the face of rapidly evolving
  browser landscape---we do not need to upgrade or rebuild our
  framework whenever there is an update of the underlying browser.  We
  achieve browser independence through selective source
  instrumentation.  % We instrument JavaScript source code selected by
  % the user of \jalangi{}.
  \emph{An attractive feature of \jalangi{} is that it can operate
    even if certain source files are not instrumented.}
\item We wanted a framework where dynamic analysis of an actual
  execution on a browser (e.g. a mobile browser) can be performed on a
  desktop or a cloud machine.  This is important when we want to
  perform a heavy-weight analysis, such as symbolic execution.  A
  heavy-weight analysis is often impossible to perform on a resource
  constrained mobile browser.  Moreover an analysis that requires
  access to various system resources, such as file system, cannot be
  implemented in a browser without significantly modifying the
  browser. \emph{We address this design constraint through a two-phase
    analysis framework.}  In the first phase, an instrumented
  JavaScript application is executed and recorded on a user selected platform
  (e.g. mobile chrome running on Android).  % The execution records
  % minimal amount of information necessary to replay the execution on a
  % different machine and sends the information to a server (or a
  % desktop computer that is running an IDE).  
  In the second phase, the recorded data is utilized to perform a user
  specified dynamic analysis on a desktop environment.
\item A dynamic analyses framework should allow easy implementation of
  a dynamic analysis.  Previous
  research~\cite{Seward:2005:UVD:1247360.1247362,Nethercote:2007:VFH:1250734.1250746,hobbs,Bond:2007:TBA:1297027.1297057,songndss05}
  and our experience with concolic testing~\cite{dart,cute} and race
  detection techniques have shown that support for shadow values and
  shadow execution could significantly simplify implementation of
  dynamic analyses techniques.  A straight-forward way to implement
  shadow value would be to replace any value, say \texttt{val}, used
  in a JavaScript execution by an object, called \emph{annotated}
  value, \texttt{\{actual: val, shadow: "tainted"\}}, where the field
  \texttt{actual} stores the actual value and the field
  \texttt{shadow} can store necessary information about \texttt{val}.
  To accomodate such replacements, we modify every operation (e.g. +,
  *, field access) performed by the JavaScript execution because every
  value, whether primitive or not, could now be wrapped by an object.
  The modified operations first retrieve the actual values from
  the annotated values representing the operands of the operation and
  then perform the operation on the actual values to compute the
  result of the operation.  % \jalangi{}
  % then calls a user-defined analysis specific function to create an
  % annotated value from the actual result of the operation.  The
  % annotated value is then returned by the modified operation.
  This simple implementation would work if we can modify every
  operation performed by a JavaScript engine.  Unfortunately,
  \jalangi{} instruments only user-specified code. Moreover,
  \jalangi{} cannot instrument native code.  Therefore, if we call
  \texttt{array.pop()}, where \texttt{array} is an annotated value and
  \texttt{pop} is a native function, we will get an exception.
  \jalangi{} alleviates this problem by using the selective
  record-replay engine: it only records the execution of the
  instrumented code and replays the instrumented code.  Any code that
  is not and can not be instrumented, including native code, is not
  executed during the replay phase.  Since \jalangi{} supports shadow
  values and shadow execution during the replay phase, it will never
  execute un-instrumented code on annotated values.  Thus, \jalangi's
  record-replay technique is necessary for correct support of shadow
  values and shadow execution.
\end{enumerate}

In \jalangi{}, we have implemented three existing dynamic analyses:

\begin{itemize}
\item Concolic testing~\cite{dart,cute}: concolic testing performs
  symbolic execution along a concrete execution path, generates a
  logical formula denoting a constraint on the input values, and
  solves a constraint to generate new test inputs that would execute
  the program along previously unexplored paths.  Our implementation
  of concolic testing supports constraints over integral, string, and
  object types and \emph{novel type constraints.}
\item Tracking origins of \texttt{null} and
  \texttt{undefined}~\cite{Bond:2007:TBA:1297027.1297057}: this
  analysis records source code locations where null and undefined
  values come into existence and reports them if they cause an error.
  Whenever there is an error due to such literals, such as accessing
  the field of a null value, the shadow value of the literal is
  reported to the user.
\item Dynamic taint
  analysis~\cite{songndss05,Clause:2007:DGD:1273463.1273490}: a
  dynamic taint analysis is a form of information flow analysis which
  checks if information can flow from a specific set of memory
  locations, called sources, to another set of memory locations,
  called sinks.  We have implemented a simple form of dynamic taint
  analysis in \jalangi{}.
\item Detecting likely type inconsistencies: this dynamic analysis
  checks if an object created at a given program location can assume
  multiple inconsistent type.  Sometimes these kind of type
  inconsistencies could point us to a potential bug in the program.
  We have noticed such issues in two SunSpider benchmark programs.
\item Simple object allocation profiler: this dynamic analysis
  computes the number of objects created at a given allocation site
  and how often the object has been accessed.  If an allocation site
  creates too many contant objects, then it could lead to memory
  inefficiency.  We have found such a problem in one of the web
  applications in our benchmark suite.
\end{itemize}

\jalangi{} is available at
\url{https://github.com/SRA-SiliconValley/jalangi} under Apache 2.0
license.  We evaluated \jalangi{} on the CPU-intensive SunSpider
benchmark suite and on several user-interaction rich web applications.
Our evaluation results show that \jalangi{} has an average overhead of
26X during recording and 30X during replay.  This is better than
PinPlay~\cite{Patil:2010:PFD:1772954.1772958} by a factor of 2X-3X and
slower than Valgrind~\cite{Nethercote:2007:VFH:1250734.1250746}.  We
also found that existing dynamic analyses could easily be implemented
in \jalangi{}.  We expect to make \jalangi{} open-source by the end of
April 2013.


\section{Technical Details}
\label{sec:technical-details}

To simplify exposition of our techniques (and to avoid explanation of
the nuances of JavaScript), we use a simple JavaScript-like imperative
language.  The syntax of this language is shown below.

{\small
\[
\begin{array}{rll}
v, v1, v2, & v3, \ldots \mbox{ are variable identifiers}\\
f, f1, f2, & \ldots \mbox{ are field identifiers}\\
p, p1, p2, & \ldots \mbox{ are function parameter identifiers}\\
\dsl{op} & \mbox{ are operators such as +, -, *, ...} \\
\mbox{Pgrm} ::= & (\ell\colon \mbox{Stmt})^*\\
\mbox{Stmt} ::= &  \dsl{var}\ v\\
  & v = c\\
  &   v1 = v2\ \dsl{op}\ v3\\
  &   v1 = \dsl{op}\ v2\\
  &   v1 = \dsl{call}(v2, v3, v4, \ldots)\\
  &   \dsl{if}\ v \ \dsl{goto}\ \ell\\
  &   \dsl{return}\ v\\
  &   v1 = v2[v3]\\
  &   v1[v2] = v3\\
  % &   v1 = \dsl{new}(v2, v3, \ldots)\\
  &   \dsl{function}\ v1(p1, \ldots) \{ (\ell\colon \mbox{Stmt})^* \} & \mbox{function definition}\\
c ::= & \mbox{number}\\
  & \mbox{string} \\
  & \dsl{undefined} \\
  & \dsl{null} \\
  & \dsl{true}\\
  & \dsl{false}\\
  & \{f1\colon v1, \ldots\} &\mbox{ object literal}\\
  & [v1, \dots] &\mbox{ array literal}\\
  & \dsl{function}\ v1(p1, \ldots) \{ (\ell\colon \mbox{Stmt})^* \} & \mbox{
    function literal}\\
\end{array}
\]
}

A program in this language is a sequence of labeled statements.  The
statements in the language are in three-address code.  $\dsl{if}\ v\
\dsl{goto}\ \ell$ is the only statement that allows conditional jump
to an arbitrary statement.  A compiler framework can be used to
convert more complex statements of JavaScript into statements of this
language by introducing temporary variables and by adding additional
statement labels.  For example, control-flow statements, such as
\dsl{while}, \dsl{for}, can be converted into a sequence of statements
in this language using $\dsl{if}\ v\ \dsl{goto}\ \ell$.  We use the
statement $v1 = \dsl{call}(v2, v3, v4, \ldots)$ to represent function,
method, and constructor calls, where $v2$ denotes the function that is
being called, $v3$ denotes the \dsl{this} object inside the function,
and $v4, \ldots$ denote the arguments passed to the function.  We use
$v1[v2]$ to denote both access to an element of an array and access to
a field of an object.

\subsection{Selective Record-Replay}
\label{sec:unopt-select-record}

We assume that the user of \jalangi{} selects a subset of the
JavaScript source in a web application for record-replay.  \jalangi{}
instruments the user-selected source for record-replay.  During the
\emph{recording} phase, the application is executed with the
instrumented files on a platform of the user's choice (e.g. a mobile
browser or a node.js interpreter).  During recording, the entire
application is executed, i.e. all instrumented and un-instrumented
JavaScript files and native codes get executed.  During the
\emph{replay} phase, \jalangi{} only replays the execution of the
instrumented sections.  This asymmetry of execution in the two phases has
two key advantages:
\begin{enumerate}
\item One could record an execution of a JavaScript application on an
  actual platform (e.g. a mobile browser) and then replay the
  execution for the purpose of debugging on a desktop JavaScript
  engine, such as node.js or a JavaScript engine embedded in an IDE.
  The replay does not require access to any browser-specific native
  JavaScript libraries such as libraries for manipulating the DOM.
\item During replay, since we avoid the execution un-instrumented code
  and native code, we can easily implement various dynamic analysis
  that depend on shadow values and shadow executions. 
\end{enumerate}

\emph{A trivial way to perform faithful record-replay of an execution
  is to record every value loaded from memory during an execution and
  use those values for corresponding memory loads in the replay
  phase.}  This approach has two challenges: 1) How do we record
values of objects and functions?  2) How do we replay an execution
when an un-instrumented function or a native function, such as the
JavaScript event dispatcher, calls an instrumented function?  Note
that we do not allow the execution of un-instrumented and native
functions during the replay phase.  Therefore, we need an alternative
mechanism to execute instrumented functions that are being invoked by
un-instrumented functions during recording.  We address the first
challenge by associating a unique numerical identifier with every
object and function and by recording the value of those unique
identifiers.  We address the second challenge by explicitly recording
and calling instrumented functions that are being invoked from
un-instrumented functions or are dispatched by the JavaScript event
dispatcher.

We avoid recording of every load of memory based on the following
observation: \emph{if we can compute the value of a memory load during
  the replay phase by solely executing the instrumented code, then we
  do not need to record the value of the load.}

In order to determine if the value of a memory load needs to be
recorded, \jalangi{} maintains a shadow memory during the recording
phase.  The shadow memory is updated along with the actual memory
during the execution of instrumented code.  Execution of
un-instrumented and native code does not update the shadow memory.
During the load of memory in the recording phase, if \jalangi{} finds
any difference between the value of the actual memory being loaded and
the value stored in the corresponding shadow memory, \jalangi{}
records the value of such memory loads. This ensures that correct
values are available during the replay phase.

\begin{figure}
{\scriptsize\[
\begin{array}{lrl}
   \dsl{var}\ v& \Longrightarrow & \dsl{var}\ v'\\
                     & & \dsl{var}\ v\\
& & \\
    v = c & \Longrightarrow & v ' = v = \Sync(c)\\
& & \\
    v1 = v2\ \dsl{op}\ v3 & \Longrightarrow & v2' = v2 = \Sync(v2, v2')\\
& & v3' = v3 = \Sync(v3, v3')\\
& & v1' = v1 = v2\ \dsl{op}\ v3\\
& & \\
    v1 = \dsl{op}\ v2 & \Longrightarrow & v2' = v2 = \Sync(v2, v2')\\
& & v1' = v1 = \dsl{op}\ v2\\
& & \\
    \dsl{if}\ v \ \dsl{goto}\ \ell & \Longrightarrow & v' = v =
    \Sync(v, v')\\
    & & \dsl{if}\ v \ \dsl{goto}\ \ell \\
& & \\
    \dsl{return}\ v & \Longrightarrow & v' = v = \Sync(v, v')\\
    & & \dsl{return}\ v \\
& & \\
    v1 = v2[v3] & \Longrightarrow & v2' = v2 =\Sync(v2, v2')\\
& & v3' = v3 = \Sync(v3, v3')\\
& & v2[v3+``'"] = v2[v3] = \\
& & \qquad\Sync(v2[v3], v2[v3+``'"])\\
& & v1' = v1 = v2[v3]\\
& & \\
    v1[v2] = v3 & \Longrightarrow & v1' = v1 = \Sync(v1, v1')\\
& & v2' = v2 = \Sync(v2, v2')\\
& & v3' = v3 = \Sync(v3, v3')\\
& & v1[v2+``'"] = v1[v2] = v3\\
& & \\
    v1 = \dsl{call}(v2, v3, v4, \ldots) & \Longrightarrow & v2' = v2 =
    \Sync(v2, v2')\\
& & v3' = v3 = \Sync(v3, v3')\\
& & v4' = v4 = \Sync(v4, v4')\\
& & \vdots \\
& & v1' = v1 = \Sync(\\
& & \quad\dsl{instrCall}(v2, v3, v4, \ldots))\\
& & \\
%     v1 = \dsl{new}(v2, v3,  \ldots) & \Longrightarrow & v2' = v2 =
%     \Sync(v2, v2')\\
% & & v3' = v3 = \Sync(v3, v3')\\
% & & \vdots \\
% & & v1' = v1 = \Sync(\dsl{I\_new}(v2, v3, \ldots))\\
% & & \\
  \{f1\colon v1, \ldots\} & \Longrightarrow &\{f1\colon v1' = v1 =\\
& &\quad  \Sync(v1, v1'), \ldots\} \\
& & \\
  \mbox{[}v1, \ldots\mbox{]} & \Longrightarrow & \mbox{[}v1' = v1 =
  \Sync(v1, v1'), \ldots\mbox{]} \\
& & \\
    \dsl{function}\ v1(p1, \ldots) \{ & \Longrightarrow &  \dsl{function}\ v1 (p1, \ldots) \{\\
\quad (\ell\colon \mbox{Stmt})^* & & \quad \Enter(v1)\\
%\} & & \quad \dsl{try}\ \{\\
\} & & \quad \dsl{var}\ p1'\\
& & \quad \vdots\\
& & \quad  (\ell\colon \mbox{Stmt})^*\\
& & \}\\
\end{array}
\]}
\caption{Instrumentation for Record-Replay. $\Sync(c)$ is
  equivalent to $\Sync(c, \dsl{undefined})$.}
\label{fig:instr2}
\end{figure}
 
Figure~\ref{fig:instr2} shows the instrumentation that \jalangi{}
performs for record-replay.  The instrumentation does not change the
behavior of the actual execution. \jalangi{} introduces a shadow
variable $v'$ for every local and global variable $v$.  \jalangi{}
introduces a local variable $p'$ for every formal parameter $p$ of an
instrumented function.  Similarly, for every field $f$ of every
object, \jalangi{} introduces a shadow field $f'$.  Note that if
$v1[v2]$ denotes access of the field denoted by the string stored in
$v2$, then $v1[v2+``'"]$ denotes the access of the corresponding
shadow field.

During the recording phase, \jalangi{} keeps the actual memory and
shadow memory in sync as much as possible.  Note that a field of an
object may not be in sync with the corresponding shadow field if the
field gets updated in native or un-instrumented code.  Whenever a
variable or a field of an object is updated, \jalangi{} adds
instrumentation to update the corresponding shadow variable or shadow
field of the object.  For example, $v1 = v2[v3]$ gets modified to $v1'
= v1 = v2[v3]$.

The instrumentation performs the following additional three
transformations:

\begin{itemize}
\item If a local or global variable $v$ or a field of an object
  $v1[v2]$ is loaded in a statement, we first call $v' = v = \Sync(v,
  v')$ or $v1[v2+"'"] = v1[v2] = \Sync(v1[v2], v1[v2+"'"])$,
  respectively, before the actual load.  In the recording phase, the
  function $\Sync$ records the value stored in the memory if the
  values stored in the actual and shadow memory are different (i.e. if
  the arguments of the $\Sync$ are different).  In the replay phase,
  $\Sync$ returns the first argument if the corresponding load in the
  recording phase was not recorded and returns the recorded value
  otherwise.  This ensures that in the replay phase, \jalangi{} gets
  the exact value that is loaded during the recording phase.
\item We replace $\dsl{call}(v2, v3, v4, \ldots)$ by
  $\Sync(\dsl{instrCall}(v2, v3, v4, \ldots))$. During the replay
  phase, function $\dsl{instrCall}$ invokes $\dsl{call}(v2, v3, v4,
  \ldots)$ if function $v2$ is instrumented.  Otherwise, it explicitly
  calls any instrumented function that is invoked while executing the
  un-instrumented or native function $v2$.  We use the function
  \texttt{replay} defined in Figure~\ref{fig:lib1} to call
  instrumented functions whose callers are not instrumented.
\item We insert the statement $\Enter(v1)$ as the first statement of
  any instrumented function with name, say $v1$.  In the recording
  phase, $\Enter(v1)$ records the value of the function $v1$.  In the
  replay phase, $\dsl{instrCall}$ invokes the recorded function if the
  function is called from a un-instrumented or native function.
\end{itemize}


Figure~\ref{fig:lib1} defines the functions $\Sync$,
$\dsl{instrCall}$, and $\Enter$, which are inserted by \jalangi{}
instrumentation.  The library maintains an array \texttt{trace} of the
recorded values along with their types.  \texttt{trace[i]} stores the
value of the $i^{\rm th}$ memory load.  The array is initialized and
populated during the recording phase and is used in the replay phase.
At the end of recording, \texttt{trace} is serialized to the
filesystem in JSON format.  During replay, the serialized file is used
to initialize \texttt{trace}.

\lstset{language=JavaScript}
\begin{figure}
 {\scriptsize 
\begin{lstlisting}[mathescape]
// persist trace after recording
// during replay initialize trace 
// from persisted trace 
var trace = []; 
var i = 0, id = 0, objectMap = [];

function getRecord(v) {
  if (v !== null && (typeof v ==='object'||  
      typeof v === 'function')){
    if (!v["*id*"]) v["*id*"] = ++id;
    return {type:typeof v, val:v["*id*"]}
  } else {
    return {type:typeof v, val:v};
  }
}

function syncRecord(rec, v) {
  var result = rec.val
  if(rec.val !==null && (rec.type==='object'|| 
      rec.type === 'function')){
     if (objectMap[rec.val])
       result = objectMap[rec.val];
     else {
      if(typeof v !== rec.type||v["*id*"]) 
       v = (rec.type==='object')?{}:function(){}
      v["*id*"] = rec.val;
      objectMap[rec.val] = v;
      result = v;
     }
  }
  return result
}

function $\Sync$(v1, v2) {
  i = i + 1;
  if (recording) {
    if (v1 !== v2)
      trace[i] = getRecord(v1);
    return v1;
  } else {
    if (trace[i])
      return syncRecord(trace[i], v1);
    else
      return v1;
  }
}

function $\Enter$(v) {
  i = i + 1;
  if (recording) {
    trace[i] = getRecord(v)
    trace[i].isFunCall = true
  }
}

function $\dsl{instrCall}$(f, o, a1, ..., an)  {
  if (recording || isInstrumented(f)) 
     return $\dsl{call}$(f, o, a1, ..., an)
  else
     return replay()
}

function replay() {
  while (trace[i+1].isFunCall) {
     var f = syncRecord(trace[i+1], undefined)
     f()
  }
  return undefined
}
\end{lstlisting}
}
  \caption{Record-Replay Library}
  \label{fig:lib1}
\end{figure}

Function $\Sync$ is defined as described before.  If the second
argument of $\Sync$ is not provided, then we assume that the second
argument is \texttt{undefined}.  \jalangi{} uses the flag
\texttt{recording} to indicate if an execution is meant for recording
or replay.  For a \texttt{recording} execution, if the two arguments
of $\Sync$ are different, then \jalangi{} records the value and the
type of the value in the sparse array \texttt{trace}.  Otherwise,
\jalangi{} skips recording, i.e. keeps the entry \texttt{trace[i]}
\texttt{undefined}.  If the value of \texttt{v1} in $\Sync$ is
restricted to primitive types (i.e. number, string, boolean,
undefined, or null), we can simply do \texttt{trace[i] = v1}.
However, the type of \texttt{v} could be an object or a function.  To
handle objects and functions, $\Sync$ calls \texttt{trace[i] =
  getRecord(v1)}, where \texttt{getRecord(v1)} returns an object whose
\texttt{type} field is set to the type of \texttt{v1} and \texttt{val}
is set to \texttt{v1} if \texttt{v1} is of primitive type.  If type of
\texttt{v1} is non-null object or function, then we use the unique
numerical id of the object or function as its value to be recorded.
The unique numerical id of a non-null object or function is stored in
its hidden field \texttt{*id*}.  If the object or function has no
unique id, \texttt{getRecord} creates and assigns a unique numerical
id to the object or function.

In a replay execution, if \texttt{trace[i]} is \texttt{undefined}
inside a call of $\Sync$, then $\Sync$ returns the value present in
the actual memory.  Otherwise, $\Sync$ returns the value recorded in
the \texttt{trace}.  $\Sync$ could simply return \texttt{trace[i]}, if
the value of \texttt{v1} in $\Sync$ is restricted to primitive types.
Since type of \texttt{v1} could be object or function,
\texttt{trace[i].type} records the type of \texttt{v1} and
\texttt{trace[i].val} stores the value or unique id of \texttt{v1} if
\texttt{v1} is of primitive type or object/function type,
respectively.  If the type of \texttt{v1} is non-null object or
function, we need to return the object or function that has the unique
id recorded in \texttt{trace[i].val}.  \texttt{sync} calls
\texttt{syncRecord(rec, v1)} to achieve this.  \texttt{syncRecord}
maintains a map, \texttt{objectMap}, from unique identifiers to
object/functions.  If \texttt{syncRecord} discovers that the recorded
unique id maps to an object/function in the \texttt{objectMap}, it
returns that object/function.  Otherwise, if \texttt{syncRecord} finds
that the recorded unique identifier has no map in the
\texttt{objectMap}, \texttt{syncRecord} does the following:
\begin{itemize}
\item If \texttt{v} is a fresh object/function (i.e. which has not
  been assigned an unique id in the current execution),
  \texttt{syncRecord} assigns the recorded unique id \texttt{rec.val}
  to the object \texttt{v} and updates \texttt{objectMap} to remember
  this mapping.  \texttt{syncRecord} returns the object \texttt{v}.
\item Otherwise, \texttt{syncRecord} has encountered an undefined
  value or a stale value.  Therefore, \texttt{syncRecord} creates a
  mock empty object/function, assigns the recorded id to that object,
  updates the \texttt{objectMap}, and returns the mock
  object/function.
\end{itemize}

The function \texttt{replay} plays an important role in the replay
phase.  It ensures that any instrumented function that got invoked
from an un-instrumented or native function, is called by \jalangi{}
explicitly.  The \texttt{replay} function is dependent on the
\texttt{enter} function inserted at the beginning of every
instrumented function.  \texttt{enter} records the value of the
function that is currently being executed.  It also sets the field
\texttt{isFunCall} of the record appended to \texttt{trace} to
\texttt{true}.  A true value of \texttt{trace[i].isFunCall} indicates
the record appended to \texttt{trace} corresponds to the invocation of
the function denoted by \texttt{trace[i].val}.  Now let us see how
this record is used in the replay phase.  \jalangi{} calls
\texttt{instrCall} in place of any \texttt{call} statement in the
code.  \texttt{instrCall}, in turn, invokes \texttt{call} if
\jalangi{} is in the recording phase, or during replay phase when
function \texttt{f} is instrumented.  This ensures that \jalangi{}
executes any function, whether instrumented or un-instrumented,
normally during the recording phase, and that \jalangi{} only executes
instrumented functions normally during the replay phase. If the
function \texttt{f} inside \texttt{instrCall} is un-instrumented, then
there is a possibility that \texttt{f} could have called some
instrumented function in the recording phase.  In order to replay the
execution of those instrumented functions, \jalangi{} calls
\texttt{replay}.  \texttt{replay} first computes the function object
by looking at the next record in the trace and then invokes it if
\texttt{isFunCall} is true.  The invocation does not pass any argument
because \jalangi{} has no record of the arguments being passed to the
function.  The arguments get synced inside the function as they are
being read inside the function.

\jalangi{} starts the replay phase by calling the \texttt{replay}
function instead of calling the entry function of the application.

 

% The addition of shadow memory and the modification of the $\Sync$
% function significantly reduces the amount of data that needs to be
% recorded during the recording phase.  Our evaluation section
% illustrates this fact.  Note that one can argue that there is no need
% to maintain shadow memory for local variables, because the values of
% local variables will be same as the value of corresponding shadow
% variables inside instrumented functions.  This is not true for
% JavaScript because a call to \texttt{eval} could change local
% variables.  Moreover, this is not true for formal parameters of a
% function because each formal parameter of a function is aliased with
% an element of the array-like object \texttt{arguments}.  One could
% perform a simple static analysis to identify the instrumented
% functions where local variables can be modified due to a call to
% \texttt{eval} or due to an access to \texttt{arguments}.  Our current
% implementation does not incorporate this optimization.

\subsection{Shadow Values and Shadow Execution}
\label{sec:shadow-values-shadow}

\jalangi{} enables a robust framework for writing dynamic program analyses
through shadow values and shadow execution.  A user-defined shadow
execution can be performed by \jalangi{} during the replay phase.
\jalangi{} only performs shadow execution of instrumented code:
without instrumentation, \jalangi{} cannot analyze the behavior of
un-instrumented or native code.

\lstset{language=JavaScript}
\begin{figure}
  
{\small 
\begin{lstlisting}[mathescape]
function AnnotatedValue(actual, shadow) {
  this.actual = actual;
  this.shadow = shadow;
}

function $\Actual$(v) {
  if (v instanceof AnnotatedValue)
    return v.actual
  return v
}

function $\Shadow$(v) {
  if (v instanceof AnnotatedValue)
    return v.shadow
  return undefined
}
\end{lstlisting}
}
  \caption{Annotated Value}
  \label{fig:annot}
\end{figure}

In shadow execution, \jalangi{} allows the replacement of any value
used in the execution by an \emph{annotated value}.  The annotated
value can carry extra information about the actual value.  For
example, an annotated value can carry taint information in a taint
analysis or a symbolic expression describing the actual value in
symbolic execution.  In \jalangi{}, we denote an annotated value using
an object of type \texttt{AnnotatedValue} defined in
Figure~\ref{fig:annot}.  An object of type \texttt{AnnotatedValue} has
two fields: the field \texttt{actual} stores the actual value and the
field \texttt{shadow} stores the shadow value, i.e. extra information
about the actual value.  A value, say $v$, in JavaScript can be
associated with shadow value, say $s$, by simply replacing $v$ by
\texttt{new AnnotatedValue}$(v, s)$.  The projection function
$\Actual(v)$ returns the actual value of $v$, if $v$ is an annotated
value and returns $v$ otherwise.  Similarly, the projection function
$\Shadow(v)$ returns the shadow value associated with $v$ if $v$ is an
annotated value and returns \texttt{undefined} otherwise.

\begin{figure}
{\scriptsize\[
\begin{array}{lrl}
   \dsl{var}\ v& \Longrightarrow & \dsl{var}\ v'\\
                     & & \dsl{var}\ v\\
& & \dsl{if} (\analysis \ \&\&\  \analysis.\usl{literal})\\
& & \quad v = \analysis.\usl{literal}(\dsl{undefined})\\
& & \\
    v = c & \Longrightarrow & v ' = v = \Sync(c)\\
& & \dsl{if} (\analysis \ \&\&\  \analysis.\usl{literal})\\
& & \quad v = \analysis.\usl{literal}(c)\\
& & \\
    v1 = v2\ \dsl{op}\ v3 & \Longrightarrow & v2' = v2 = \Sync(v2, v2')\\
& & v3' = v3 = \Sync(v3, v3')\\
& & v1' = v1 = \Actual(v2)\ \dsl{op}\ \Actual(v3)\\
& & \dsl{if} (\analysis \ \&\&\  \analysis.\usl{binary})\\
& & \quad v1 = \analysis.\usl{binary}(\dsl{op}, v2, v3, v1)\\
& & \\
    v1 = \dsl{op}\ v2 & \Longrightarrow & v2' = v2 = \Sync(v2, v2')\\
& & v1' = v1 = \dsl{op}\ \Actual(v2)\\
& & \dsl{if} (\analysis \ \&\&\  \analysis.\usl{unary})\\
& & \quad v1 = \analysis.\usl{unary}(\dsl{op}, v2, v1)\\
& & \\
    \dsl{if}\ v \ \dsl{goto}\ \ell & \Longrightarrow & v' = v =
    \Sync(v, v')\\
& & \dsl{if} (\analysis \ \&\&\  \analysis.\usl{conditional})\\
& & \quad \analysis.\usl{conditional}(v)\\
    & & \dsl{if}\ \Actual(v) \ \dsl{goto}\ \ell \\
& & \\
    \dsl{return}\ v & \Longrightarrow & v' = v = \Sync(v, v')\\
    & & \dsl{return}\ v \\
& & \\
    v1 = v2[v3] & \Longrightarrow & v2' = v2 =\Sync(v2, v2')\\
& & v3' = v3 = \Sync(v3, v3')\\
& & \Actual(v2)[\Actual(v3)+``'"] = \Actual(v2)[\Actual(v3)] = \\
& & \ \ \Sync(\Actual(v2)[\Actual(v3)], \Actual(v2)[\Actual(v3)+``'"])\\
& & v1' = v1 = \Actual(v2)[\Actual(v3)]\\
& & \dsl{if} (\analysis \ \&\&\  \analysis.\usl{getField})\\
& & \quad v1 = \analysis.\usl{getField}(v2, v3, v1)\\
& & \\
    v1[v2] = v3 & \Longrightarrow & v1' = v1 = \Sync(v1, v1')\\
& & v2' = v2 = \Sync(v2, v2')\\
& & v3' = v3 = \Sync(v3, v3')\\
& & \Actual(v1)[\Actual(v2)+``'"] = \Actual(v1)[\Actual(v2)] = v3\\
& & \dsl{if} (\analysis \ \&\&\  \analysis.\usl{putField})\\
& & \quad \Actual(v1)[\Actual(v2)] = \\
& & \qquad\qquad \analysis.\usl{putField}(v1, v2, v3)\\
& & \\
    v1=& \Longrightarrow & v2' = v2 =
    \Sync(v2, v2')\\
\quad \dsl{call}(v2, v3, v4, \ldots) & & v3' = v3 = \Sync(v3, v3')\\
& & v4' = v4 = \Sync(v4, v4')\\
& & \vdots \\
& & v1' = v1 = \Sync(\\
& & \quad\dsl{instrCall}(\Actual(v2), v3, v4, \ldots))\\
& & \dsl{if} (\analysis \ \&\&\  \analysis.\usl{call})\\
& & \quad v1 = \analysis.\usl{call}(v2, v3, v4, \ldots, v1)\\
& & \\
  \{f1\colon v1, \ldots\} & \Longrightarrow &\{f1\colon v1' = v1 =\\
& &\quad  \Sync(v1, v1'), \ldots\} \\
& & \\
  \mbox{[}v1, \ldots\mbox{]} & \Longrightarrow & \mbox{[}v1' = v1 =
  \Sync(v1, v1'), \ldots\mbox{]} \\
& & \\
    \dsl{function}\ v1(p1, \ldots) \{ & \Longrightarrow &  \dsl{function}\ v1 (p1, \ldots) \{\\
\quad (\ell\colon \mbox{Stmt})^* & & \quad \Enter(v1)\\
%\} & & \quad \dsl{try}\ \{\\
\} & & \quad \dsl{var}\ p1'\\
& & \quad \vdots\\
& & \quad  (\ell\colon \mbox{Stmt})^*\\
& & \}\\
\end{array}
\]}
\caption{Instrumentation for Record-Replay and Shadow Execution}
\label{fig:instr3}
\end{figure}

If a JavaScript value is replaced by a user-defined annotated value
during an analysis, the built-in JavaScript operations will fail.  For
example, if we replace the number value $53$ by the annotated value
\texttt{new AnnotatedValue}$(53, null)$, then addition of this value
with another number, say $31$, would result in \texttt{NaN} instead of
$84$.  To avoid such situations, we instrument code so that \jalangi{}
performs the built-in JavaScript operations on the actual values
instead of the annotated values.  For example, $v1\ \dsl{op}\ v2$ is
replaced by $\Actual(v1)\ \dsl{op}\ \Actual(v2)$.  Similarly, $v1[v2]$
is replaced by $\Actual(v1)[\Actual(v2)]$.  The instrumentation
inserted by \jalangi{} to perform shadow execution with shadow values
along with record-replay is shown in Figure~\ref{fig:instr3}.

\lstset{language=JavaScript}
\begin{figure}
 {\small 
\begin{lstlisting}[mathescape]
function syncRecord(rec, tv) {
M: var v = $\Actual$(tv), result = rec.val
  if(rec.val !==null && (rec.type==='object'|| 
      rec.type === 'function')){
     if (objectMap[rec.val])
       result = objectMap[rec.val];
     else {
      if(typeof v !== rec.type||v["*id*"]) 
       v = (rec.type==='object')?{}:function(){}
      v["*id*"] = rec.val;
      objectMap[rec.val] = v;
      result = v;
     }
  }
M: if ($\Actual$(tv) === result)
M:  result = tv
  return result
}
\end{lstlisting}
}
  \caption{Updated syncRecord for Shadow Execution.  Modified lines
    are labeled with \texttt{M:}\vspace*{1ex}}
  \label{fig:lib3}

\end{figure}

The instrumentation assumes that the global variable $\analysis$ could
point to an user-defined analysis object during the replay phase.
After the execution of a JavaScript statement, the corresponding
method in the $\analysis$ object is called to perform a user-specific
analysis.  For example, consider the statement $v1 = v2\ \dsl{op}\
v3$.  After the execution of this statement in the replay phase,
\jalangi{} calls the $v1 = \analysis.\usl{binary}(\dsl{op}, v2, v3,
v1)$ to perform an analysis specific function for the binary operation
$\dsl{op}$.  For example, if $v2$ is the number $53$ and $v3$ is the
annotated value \texttt{new AnnotatedValue}$(31,$``tainted''), then
after the execution of the actual statement $v1$ will be $84$ and then
execution of $v1 = \analysis.\usl{binary}(\dsl{op}, v2, v3, v1)$ could
store \texttt{new AnnotatedValue}$(84,$``tainted'') in $v1$ to
represent the fact if one of the operands of a binary operation is
tainted, then the result of the operation is also tainted.  Following
is another example in the context of symbolic execution.  If $v2$ is
the annotated value \texttt{new AnnotatedValue}$(1, "2x_1+1")$ and
$v3$ is the annotated value \texttt{new AnnotatedValue}$(3,
"x_2-x_1")$, then after the execution of $v1 =
\analysis.\usl{binary}(+, v2, v3, v1)$, where $\analysis$ performs
symbolic execution, $v1$ will be the annotated value \texttt{new
  AnnotatedValue}$(4, "x_1+x_2")$.  Note that in the symbolic
execution, the symbolic expression corresponding to a concrete value
is represented as a string in the shadow value.

\subsection{Example Analysis: Tracking Origin of null and undefined Values}
\label{sec:example-analysis}

In Figure~\ref{fig:lib3} we describe a simple dynamic analysis using
the shadow execution framework for \jalangi{}.  The analysis tracks
the origin of \texttt{null} and \texttt{undefined} in a JavaScript
execution.  If during an execution, access is made to the field of a
\texttt{null} or \texttt{undefined} value, or if an invocation of a
value which is \texttt{null} or \texttt{undefined} is encountered, the
analysis could report the line number of code where the \texttt{null}
or \texttt{undefined} value originated.

\begin{figure}
 {\small 
\begin{lstlisting}
anlys = {
  literal: function(c) {
    if (c === null || c === undefined) {
      return new AnnotatedValue(c,getLocation())
    }
  },

  getField: function(v1, v2, r) {
    if (r === null || r === undefined){
      return new AnnotatedValue(r,getLocation())
    }
  },

  call: function(f, o, a1,...,an, r) {
    if (r === null || r === undefined){
      return new AnnotatedValue(r,getLocation())
    }
  }
}
\end{lstlisting}
}
  \caption{Tracking Origins of undefined and null}
  \label{fig:lib3}
\end{figure}

The analysis creates an object $\analysis$, where we define the
methods $\usl{literal}$, $\usl{getField}$, and $\usl{call}$.  The
operations corresponding to these methods could create \texttt{null}
and \texttt{undefined} values.  Therefore, if the value returned by
any of these operations is \texttt{null} or \texttt{undefined}, we
annotate the return value with the location information.
\texttt{getLocation()} returns the line number in the original code
where the instrumentation was inserted by \jalangi{}.

The above example shows how one could implement a dynamic analyses
using \jalangi{}.  In our framework, we have implemented full concolic
testing and taint analysis using shadow execution.  We believe that
many other dynamic analyses could be implemented easily using
\jalangi{}.

\section{Example}
\label{sec:example}

\begin{figure}
 {\scriptsize
\begin{lstlisting}[mathescape]
// un-instrumented
function foo() {
  mydoc = document;
}
// to be instrumented
var mydoc;

function myapp() {
  document.onload = function myload () {
    var url = document.URL;
    foo();
    return mydoc;
  }
}();

trace = [
    // sync function literal myapp and 
    // set myapp["*id*"] = 1  
 {type: "function", val: 1},
    // record enter(myapp)
 {type: "function", val: 1, isFunCall: true}, 
    // sync load of document and 
    // set document["*id*"] = 2
 {type: "object", val: 2}, 
    // sync function literal myload 
    //and set myload["*id*"]= 3 
 {type: "function", val: 3}, 
    // record enter(myload) 
    // where myapp is called by the event dispatcher  
 {type: "function", val: 3, isFunCall: true}, 
    // sync getField, document["URL"]
 {type:"string",val:"http://127.0.0.1/index.html"}, 
    // sync function literal foo and 
    // set foo["*id*"]= 4
 {type: "function", val: 4}, 
    // sync load of mydoc on return
 {type: "object", val: 2} 
]
\end{lstlisting}
}
  \caption{An example JavaScript program.  Assume that the function
    \texttt{foo} is not instrumented.  Executing the program on a
    browser generates the \texttt{trace}.  }
  \label{fig:example}
\end{figure}

Consider the example JavaScript program in Figure~\ref{fig:example}.
Let us assume that the entire program is instrumented except the body
of the function \texttt{foo}.  The trace generated by an execution of
the program in a browser is also shown in the Figure.  Note that
during the recording phase, we create an unique identifier for each of
the objects accessed inside the body of the program.  The object
\texttt{document} is available in the browser, but the object never
got created in the body of the program.  During the replay, a mock
object is created for \texttt{document} and
\texttt{document[}``\texttt{*id*}''\texttt{]} is set to 2, an identifier obtained from
the recorded trace.  \texttt{document.URL} is set to
\texttt{"http://127.0.0.1/index.html"}, a string value obtained from
the trace.  During the recording phase, \texttt{mydoc} gets set to
document inside un-instrumented code.  Therefore, after the execution
of \texttt{foo}, \texttt{mydoc} will contain the object document and
the shadow variable \texttt{mydoc'} will still be \texttt{undefined}.
\jalangi{} will, therefore, record the value of \texttt{mydoc}, when
it is returned from \texttt{myload}.  During the replay, \jalangi{}
will sync the value of \texttt{mydoc}, which it will discover in
\texttt{objectMap}.  The value of \texttt{mydoc} will be set to the
mock object with id 2 created during the replay.  Thus the replay
phase will faithfully mimic the recorded execution even in a
non-browser environment.

\section{Implementation}
\label{sec:implementation}

We have implemented \jalangi{} in JavaScript.  The code of this
framework is available under Apache 2.0 open-source license at
\url{https://github.com/SRA-SiliconValley/jalangi}.  In the actual
implementation, we do not transform JavaScript into the three-address
code described in Section~\ref{sec:technical-details}.  Rather we
modify the AST in place by replacing each operation with an equivalent
function call.

\subsubsection*{Handling \texttt{eval}}
\label{sec:handling-texttteval}

\jalangi{} exposes the instrumentation
library as a function \texttt{instrumentCode}.  This enables us also
to dynamically instrument any code that is created and evaluated at
runtime.  For example, we modify any call to \texttt{eval(s)} to
\texttt{eval(instrumentCode(s))}.

\subsubsection*{Handling Exceptions}
\label{sec:handling-exceptions}

Exceptions do not pose any particular challenge in \jalangi{} except
for uncaught exceptions being thrown from un-instrumented code.  We
wrap every function within a try-catch-finally block.  In the catch
block, we re-throw the exception.  In the finally block, we call any
analysis specific code corresponding to the function call.

\subsubsection*{Handling AJAX Calls and Event Handlers}
\label{sec:handling-exceptions}

Event handlers and handlers of AJAX calls appear as top-level function
invocations in the recorded trace.  If the handlers are instrumented,
then the \texttt{replay} function defined in Figure~\ref{fig:lib1}
invokes them in the order in which they were invoked in the recorded
execution. 

In record-replay described in Figure~\ref{fig:instr2}, we record any
literal value, any value returned by a function call, and any function
value that is executed.  This could still result in large amount of
record data.  In our implementation, we avoid recording any literal
value.  We only record the return value of a function, if the function
is un-instrumented or native.  Similarly, we avoid recording a
function value at the beginning of the execution of the function, if
the function is called from an instrumented function.

\subsubsection*{Concolic Testing}
\label{sec:concolic-testing}

We have implemented concolic testing as an analysis in \jalangi{}.
We store the symbolic expression corresponding to each concrete value
in its shadow value.  Concolic execution takes place during the replay
phase: the shadow execution updates the shadow value of each
value.

In our implementation of concolic testing, we handle linear integer
constraints and string constraints involving concatenation, length,
and regular expression matching.  We also handle type constraints and
a limited set of constraints over pointers.  For example, if the type
of an input variable is unknown, we infer the possible types of the
variable by observing the operations performed on the variable.

\subsubsection*{Dynamic Taint Analysis}
\label{sec:taint-analysis}

A dynamic taint analysis is a form information flow analysis which
checks if information can flow from a specific set of memory
locations, called sources, to another set of memory locations, called
sink.  We have implemented a simple form of dynamic taint analysis in
\jalangi{}.  In the analysis, we treat read of any field of any
object, which has not previously been written by the instrumented
source, as a source of taint.  We treat any read of a memory location
that could change the control-flow of the program as a sink.  We
attach taint information with the shadow value of an actual value.
Taint information is propagated by implementing the various operations
in the analysis.  For example, if any of the operands of an operation
is tainted, then we return an annotated value which is marked as
tainted.

\subsubsection*{Detecting Likely Type Inconsistencies}
\label{sec:detect-likely-type}

The dynamic analysis checks if an object/function created at a given
program location can assume multiple inconsistent types.  It computes
the types of object and function values created at each definition
site in the program.  Specifically, the analysis associates every
object/function value with the static program location where the
object/function value got created.  It also maps each such program
location to the type that the objects/functions created at the
location can assume during the course of the execution.  If an object
or a function value defined at a program location has been observed to
assume more than one type during the execution, the analysis reports
the program location along with the observed types.  Sometimes these
kind of type inconsistencies could point us to a potential bug in the
program.  We have noticed such issues in two SunSpider benchmark
programs.

\subsubsection*{Simple Object Allocation Profiler}
\label{sec:simple-object-alloc}

This dynamic analysis records the number of objects created at a given
allocation site and how often the fields of the objects created at a
given allocation site has been accessed.  The analysis also tracks if
the objects' fields have been updated, that is the analysis tracks if
the objects created at a given allocation site are read-only or a
constant.  The analysis reports the maximum and average difference
between the object creation time and the most recent access time of
the object .  Time is reported in terms of the number of instructions
being executed.  If an allocation site creates too many contant
objects, then it could lead to performance issues.  We have found such
an issue in one of the web applications in our benchmark suite.


\section{Evaluation}
\label{sec:evaluation}

We next report our results of evaluating \jalangi{} on several
benchmark programs.  In our evaluation, we focussed on four aspects:
1) ease of writing dynamic analyses, 2) fidelity and robustness of
record-replay, 3) performance of \jalangi{}, and 4) programming issues
detected during dynamic analyses.

\subsection{Ease of Writing Dynamic Analyses}
\label{sec:ease-writing-dynamic}

We have written three dynamic analyses and a condition coverage tool
on top of \jalangi{}.  The condition coverage tool has 47 lines of
JavaScript code, the origin tracker for null and undefined has 61
lines of JavaScript code, taint analysis has 68 lines of code, and
concolic testing has 2225 lines of code.  In comparison, a concolic
testing tool for Java with lesser functionalities had more than 20,000
lines of code.  Even though number of lines of code is not a good
measure for the ease of writing a dynamic analysis, it provides a
rough estimate of the complexity of writing an analysis on top of
\jalangi{}.  We believe that \jalangi{}'s support for shadow values
and shadow execution in the form of a simple $\analysis$ API
significantly reduces the barrier to implement various dynamic
analyses.  An implementor of a dynamic analysis need not worry about
the quirks and nuances of JavaScript.  In future, we plan to enrich
\jalangi{} with several other dynamic analyses.

\subsection{Fidelity and Robustness}
\label{sec:fidelity-robustness}

By fidelity, we mean the similarity between recording and replay
executions.  By robustness, we mean the ability of \jalangi{} to
handle a program without introducing any errors or exceptions of its
own.  To check fidelity of \jalangi{}, we recoded all memory loads
both in record and replay phases and checked if the two sequences of
loads are the same.  We also recorded the execution paths taken by
both record and replay phases and checked if they are the same.  To
check robustness, we ran \jalangi{} on several real-world programs.

We managed to run \jalangi{} without any error on all programs that we
considered for evaluation.  This includes the SunSpider benchmark
suite for JavaScript and several web apps developed for the Tizen OS.
We list these benchmarks in the next section.  We also observed that
the record and the corresponding replay executions of these benchmarks
in \jalangi{} produced exactly the same sequence of memory loads and
followed exactly the same execution paths.

\subsection{Performance of  JALANGI}
\label{sec:performance-jalangi}

\begin{table*}
\begin{minipage}{0.6\textwidth}
{\scriptsize
\begin{center}
\begin{tabular}{|l|r|r|r|r|r|r|r|} \hline
\multirow{2}{*}{Benchmark} & \multirow{2}{*}{LOC} & \scriptsize{Records} & \scriptsize{fLoads}& \scriptsize{SlowR} &
\multicolumn{3}{|c|}{\scriptsize{ Slowdown in Replay}}\\
& & & & & empty & taint & track \\
\hline
3d-cube & 339 & 3670 & 0.09 & 18.33  & 25.16 & 28.67 & 26 \\
3d-morph& 56 & 6 & < 0.01 & 18.2 & 33.2 & 35.83 & 33.6 \\
3d-raytrace& 443 & 79791 & 2.68 & 38.17 & 29.05 & 30.5 & 35\\
b-trees& 52 & 146048 & 18.26 & 57.8 & 40 & 42.4 & 42.8\\
fannkuch& 68 & 246 & < 0.01 & 40.6 & 76.4 & 73 & 80.4 \\
nbody& 170 & 78 & < 0.01 & 19 & 25.8& 25.67 & 24.16\\
nsieve& 39 & 5 & < 0.01 & 16.4 & 23.6 & 30 & 24.2\\
3bit-in-byte& 38 & 1 & < 0.01& 16.6& 29 & 31 & 30.2 \\
bits-in-byte& 26 & 1 & < 0.01& 25 & 25 & 51.4 & 47 \\
bitwise-and& 31 & 1 & < 0.01& 12.83 & 21.83 & 29.2 & 26.2\\
controlflow& 25 & 1 & < 0.01& 20 & 33.2 & 34.6 & 28.33\\
crypto-md5& 288 & 42 & < 0.01& 12 & 18 & 22.2 & 22\\
crypto-sha1& 225 & 52 & < 0.01& 13.4& 19.4 & 21 & 21.2\\
date-tofte& 300 & 32018 & 1.59 & 92.16 & 92.67 & 92.83 & 95.5\\
date-xparb& 418 & 95715 & 17.81 & 29.83 & 21 & 22.67 & 25.67\\
math-cordic& 101 & 8 & < 0.01 & 29.6 & 35.6 & 45.4 & 40.17\\
partial-sums& 33 & 5 & < 0.01& 14.6 & 23.4 & 22.16& 23.8\\
spectral-norm& 51 & 15 & < 0.01& 19.8& 25.2 & 29.2 & 29.4\\
regexp-dna& 1714 & 42 & 21 & 2 & 4 & 3.17 & 3.8\\
string-fasta& 90 & 56947 & 2.77 & 40.17 & 30.33 & 34.5 & 38.6\\
string-tagcloud& 266 & 117577 & 16.23 & 51.42 & 50.86 & 44 & 42.8\\
string-unpack& 67 & 193057 & 33.21 & 29.88 & 13.25 & 13.75 & 17\\
nsieve-bits& 35 & 3 & < 0.01 & 20 & 36.6 & 45.4 & 40 \\
crypto-aes& 425 & 23926 &0.73  & 19 & 21 & 23.67 & 23 \\
string-validate& 90 & 60 & 13.27 & 1.5 & 1.5 & 1.4 & 1.5\\
string-base64& 136 & 40965 & 3.38 & 25 & 27.2 & 29.6 & 29.2\\
\hline 
annex& 9663 & 87623 & 0.86 & - & - & - & - \\
calculator& 787 & 1288 & 17.64 & - & - & - & - \\
 go& 10,039 & 114609 & 0.97 & - & - & - & - \\
tenframe& 1491 &4656 & 28.89 & - & - & - & - \\
shopping& 5397 & 1144 & 22.79 & - & - & - & - \\
\hline 
\end{tabular}
\end{center}}
\caption{Results: ``Records'' column reports number of values of recorded,
``fLoads'' reports \% of loads that were recorded, ``SlowR'' reports
slowdown during recording compared to normal execution.}
\label{tab:results}
\end{minipage}
\begin{minipage}{0.38\textwidth}
\begin{tabular}{|r|}
\hline\\
{\scriptsize
\begin{lstlisting}[mathescape]
function isValidQuery(str)
{
// (1) check that str contains "/" followed
// by no "/" and containing "?q=..."
 var slash = str.lastIndexOf('/');
 if (slash < 0){
   return false;
 }
 var rest = str.substring(slash + 1);
 if(!(RegExp('\\\?q=[a-zA-Z]+')).test(rest)){
   return false;
 }
// (2) Check that str starts with "http://"
 if (str.indexOf("http://")!==0){
   return false;
 }
// (3) Take the string after "http://"
// strip the "www." off if present
 var t=str.substring("http://".length,slash);
 if (t.indexOf("www.")===0){
   t = t.substring("www.".length);
 }
// (4) Check that the rest is either
// "live.com" or "google.com"
 if(t !=="google.com" && t!=="live.com"){
   return false;
 }
 // str survived all checks
 return true;
}

\end{lstlisting}
}\\
\hline
\end{tabular}
\caption{Sample code for evaluating performance of concolic testing}
\label{tab:conc}
\end{minipage}
\end{table*}

We performed record-replay on 26 programs in the JavaScript SunSpider
(\url{http://www.webkit.org/perf/sunspider/sunspider.html}) benchmark
suite and on five web apps written for the Tizen OS using
HTML5/JavaScript
(\url{https://developer.tizen.org/downloads/sample-web-applications}).
The web apps include \emph{annex}---a two-player strategy game,
\emph{shopping list}---which uses local storage API of HTML5,
\emph{scientific calculator}, \emph{go}---a two-player strategy game,
and \emph{tenframe}---a math-based three-game combo for kids.  During
the replay phase of these benchmark programs, we ran three dynamic
analyses: no analysis (denoted by \emph{empty}), tracking origins of
null and undefined (denoted by \emph{track}), and a taint analysis
(denoted by \emph{taint}).  We report the overhead associated with the
recording and replay phases in Table~\ref{tab:results}.  We also
report the number of values we recorded for each benchmark program and
the number of values that were loaded from the memory during the
execution.  The experiments were performed on a laptop with 2.3 GHz
Intel Core i7 and 8 GB RAM.  We ran the web apps on Chrome 25 and
performed the replay executions on node.js 0.8.14.

The SunSpider benchmarks have relatively small number of lines of
code, but they perform CPU intensive computations.  The web apps
perform both CPU intensive computations and manipulation of the DOM.
We didn't measure the slowdown of the web apps because these are
mostly interactive applications.  For the SunSpider benchmark suite,
we observed an average slowdown of 26X during the recording phase with
a minimum of 1.5X and a maximum of 93X.  On the \emph{empty} analysis
during the replay phase, we observed an average slowdown of 30X with a
minimum of 1.5X and a maximum of 93X.  \emph{Track} analysis showed an
average slowdown of 32.75X with a minimum of 1.5X and a maximum of
96X.  The slowdown in recording is 2X-3X lower than that of
PinPlay~\cite{Patil:2010:PFD:1772954.1772958} and the slowdown in the
analysis phase is slightly higher than slowdown noticed in
valgrind~\cite{Nethercote:2007:VFH:1250734.1250746}, a heavy-weight
dynamic analysis tool for x86.  We didn't make any effort to optimize
our implementation, but we believe suitable optimizations could
further reduce the overhead.  For some programs in the SunSpider suite
we noticed that the number values recorded is quite high and recording
phase has higher overhead than replay.  This because these programs
made many expensive native calls.  The return values of those calls
were recorded.  Replay skipped the execution of those native calls, so
we noticed lower overhead for replay.

In \jalangi{}, if we record every memory load, then we notice a
slowdown of 300X -1000X.  Our proposed use of shadow memory
significantly reduces the number of loads that we had to record for a
faithful replay.  The column titled ``\% of Loads Recorded'' reports
the reduction in percentage.  We noticed an average reduction of
6.52\% and a median reduction of 0.73\%.  Programs doing a lot of
native calls and performing frequent manipulation of the DOM resulted
in large recoding of memory loads.

% We performed concolic testing on a number of small benchmarks
% involving integers, strings, and type constraints.  The slowdown
% noticed in these benchmarks was mostly determined by the cost of
% invoking the SMT solver.

Based on our evaluation, we are optimistic about the utility of
\jalangi{} as a tool framework aiding web developers.  We believe that
the utility offered by \jalangi{} is much more valuable compared to
the additional performance penalty that the developers
observe. Moreover, this additional penalty would be incurred only
during the development phase, and the instrumentation introduced by
\jalangi{} would not become a part of the actual applications deployed
to users.

\subsection{Performance of concolic testing}
\label{sec:conc-test-small}

We ran concolic testing on several programs ported from a concolic
testing engine for Java.  Even though concolic testing is not the focus of
this paper, we report the results of running concolic testing on a
small program (shown in Table~\ref{tab:conc}), which has complex
string operations involving integers, string length, regular
expression matching, and concatenation.  This program is a slight
variant of the program used as a case study
in~\cite{Bjorner:2009:PFA:1532891.1532927}.  In concolic testing, we
only use the theory of linear integers of CVC3~\cite{BT07} and model
string operations using this theory.  For this program, we generated 9
input strings corresponding to the 9 distinct execution paths of the
program.  We noticed an average slowdown of 145X during concolic
execution with a maximum slowdown of 613X and a minimum slowdown of
1.4X.  The recording phases showed a slowdown of 1.2X.  The slowdown
in the concolic execution phase is mostly due to the calls to the SMT
solver.

\subsection{Issues Detected by Dynamic Analyses}
\label{sec:issu-detect-dynam}

The likely type inconsistency checker noticed that the function
\texttt{safe\_add(x, y)} (shown below) in \texttt{crypto-sha1.js} of the SunSpider
benchmark suite is mostly called with both of its arguments set to number,
but at one location it was invoked with the second argument set to
\texttt{undefined}.   We believe that this could be an unintended
behavior.   

{\scriptsize
\begin{lstlisting}[mathescape]
function safe_add(x, y)
{
  var lsw = (x & 0xFFFF) + (y & 0xFFFF);
  var msw = (x >> 16) + (y >> 16) + (lsw >> 16);
  return (msw << 16) | (lsw & 0xFFFF);
}
\end{lstlisting}
}

The likely type inconsistency checker reported that the function
\texttt{CreateP} (shown below) in \texttt{3d-cube.js} of the SunSpider
benchmark suite is mostly called as a constructor, but at one location
it was invoked as a function.  As result of the function call, the
program creates an unnecessary \texttt{V} field in the global object.
We believe that this call is a possible programming error.

{\scriptsize
\begin{lstlisting}[mathescape]
function CreateP(X,Y,Z) {
  this.V = [X,Y,Z,1];
}
\end{lstlisting}
}
 
The object allocation profiler noticed that the method
\texttt{getValue(place, \_board)} in the Annex game webapp creates a
constant object containing at least 64 numbers thousands of times.  We
believe that such unnecessary creation of the constant object can be
avoided by hoisting the object creation outside the method.

\section{Related Work}
\label{sec:related-work}

There is a large body of work on record-replay systems
(see~\cite{Cornelis03ataxonomy,Dionne96ataxonomy} for survey of this
area).  In this section, we discuss the papers that are closely
related to \jalangi{}.

JSBench~\cite{Richards:2011:ACJ:2048066.2048119} is a technique for
creating JavaScript benchmarks using record-replay mechanisms.
JSBench captures the interaction of an web application with its
surrounding execution environment.  It then creates a replayable
packaged JavaScript benchmark which can execute in the absence of the
surrounding environment.  JSBench captures the arguments passed and
value returned from external function calls.  It also captures field
accesses by external components.  However, JSBench does not capture
all memory loads or memory loads that could potentially be modified by
eval or un-instrumented code.  Therefore, JSBench could function
improperly in the presence of un-instrumented code.  \jalangi{}
alleviates this problem by maintaining shadow memory. % Another
% important differentiator lies in the ability of \jalangi{} to handle
% closures effectively. Most of the real world applications employ
% closures, and closures are a frequent cause for developer agony. Since
% \jalangi{} employes shadow memory mechanism, it has the ability to
% support closures, unlike JSBench.

PinPlay~\cite{Patil:2010:PFD:1772954.1772958}, built on top of dynamic
instrumentation framework PIN~\cite{Luk:2005:PBC:1065010.1065034} for
x86, uses ideas similar to shadow
memory~\cite{Narayanasamy:2006:ALO:1140277.1140303} to reduce the
number of memory logs.  PinPlay keeps shadow memory, which they call
UserMem, in sync with the actual memory at the byte and word level.
%This requires them to keep track of entire memory used by the program.
In JavaScript it is not possible to keep track of memory at byte and
word level.  \jalangi{} uses a novel technique based on unique
identifiers to record and sync objects and functions and uses mock
objects to mimic behaviors of objects created outside instrumented
code.  
% Since \jalangi{} does not track memory at byte or word level,
% \jalangi{} is more efficient than PinPlay.

Mugshot~\cite{Mickens:2010:MDC:1855711.1855722} is another
record-replay system for JavaScript that captures all events in a
JavaScript program and allows developers to deterministically replay
past executions of web applications.
Ripley~\cite{Vikram:2009:RAS:1653662.1653685} replicates execution of
a client-side JavaScript program on a server side replica to
automatically preserve the integrity of a distributed computation.
DoDOM~\cite{Pattabiraman:2010:DLD:1913797.1914375} records user
interaction sequences with web applications repeatedly executes the
application under the captured sequence of user actions and observes
its behavior. % Based on the observations, DoDOM extracts a set of
% invariants on the web application's DOM structure.

The idea of shadow values in the context of x86 binaries has been
previously proposed
in~\cite{Nethercote:2007:VFH:1250734.1250746,Zhao:2010:1772954.1772960}
and has been used in several analysis
tools~\cite{Zhao:2010:1772954.1772960,songndss05,Bond:2007:TBA:1297027.1297057,hobbs}.
Instead of creating a separate address space for shadow values,
\jalangi{} wraps each JavaScript value in an object of type
\texttt{AnnotatedValue}.  This simple technique is possible due to the
dynamic nature of JavaScript.

In the recent years, several
static~\cite{Yu:2007:JIB:1190216.1190252,Jensen:2010:IAL:1882094.1882114,Anderson:2005:TTI:2144892.2144917,manuicse13,Wei:2012:BAJ:2384716.2384758,Sridharan:2012:CTP:2367163.2367191}
and dynamic
analyses~\cite{Petrov:2012:RDW:2254064.2254095,Richards:2010:ADB:1806596.1806598,Artzi:2011:FAT:1985793.1985871,Mesbah:2009:IAT:1555001.1555037}
tools for JavaScript have been proposed.  Richards et
al.~\cite{Richards:2010:ADB:1806596.1806598} observed that dynamic
features are widely used in JavaScript programs.  These dynamic
features make static analysis of JavaScript applications hard and
previous research efforts have either ignored or made incorrect
assumptions regarding these dynamic features.  Dynamic analysis tools
developed for JavaScript include tools for
testing~\cite{Artzi:2011:FAT:1985793.1985871,Saxena:2010:SEF:1849417.1849985},
race detection~\cite{Petrov:2012:RDW:2254064.2254095}, and security
analysis~\cite{Vikram:2009:RAS:1653662.1653685}.  However, there
exists no dynamic analysis framework for JavaScript similar to
valgrind~\cite{Nethercote:2007:VFH:1250734.1250746},
PIN~\cite{Luk:2005:PBC:1065010.1065034},
DynamoRIO~\cite{Bruening:2003:IAD:776261.776290} for x86.  \jalangi{}
tries to fill this gap by providing a dynamic analysis framework in
which one could easily prototype and build sophisticated
browser-independent dynamic program analyses for JavaScript.

{\small
\bibliographystyle{abbrv}
\bibliography{jalangi}
}
\end{document}

%UserMem is always updated with the
%logged value to avoid duplicate logging in the future.