\documentclass{sig-alternate}

\usepackage{xspace}
\usepackage{algorithmic}
\usepackage{algorithm}
\usepackage{listings}
\usepackage{float}
\usepackage{soul}
\usepackage{multirow}
\usepackage{url}

\floatstyle{boxed}
\restylefloat{figure}
\setlength{\floatsep}{0pt}
\renewcommand{\topfraction}{0.95}
\renewcommand{\textfraction}{0.05}
\renewcommand{\floatpagefraction}{0.95}

\def\jalangi{\textsc{Jalangi}}
\newcommand \dsl [1] {\ensuremath{{\tt #1}}\xspace}
\newcommand \usl [1] {\mbox{\underline{\tt #1}}\xspace}
\newcommand \Sync{\dsl{sync}}
\newcommand \Actual{{\tt getConcrete}}
\newcommand \Shadow{{\tt getSymbolic}}
\newcommand \Enter{\dsl{enter}}
\newcommand \Exit{\dsl{exit}}
\newcommand \analysis{\usl{anlys}}

\lstdefinelanguage{JavaScript}{
  keywords={typeof, new, true, false, catch, function, return, null, catch, switch, var, if, in, while, do, else, case, break},
  keywordstyle=\bf\tt,
  ndkeywords={anlys, literal, binary, unary, putField,
    getField, conditional, call},
  ndkeywordstyle=\underline,
  sensitive=false,
  comment=[l]{//},
  morecomment=[s]{/*}{*/},
  commentstyle=\sl,
  stringstyle=\sf,
  morestring=[b]',
  morestring=[b]",
literate={\$}{{\textcolor{blue}{\$}}}1
}

\lstdefinelanguage{pseudo} { 
  	basicstyle=\tt, 
	keywordstyle=\bf\tt,
  	morekeywords= { Object, foreach, to, for, new, while, do, if, then, else, return, caoea, int, in  }, 
    sensitive=false, 
    morecomment=[l]{//}, 
    morecomment=[s]{/*}{*/},
	morestring=[b]"
}

\begin{document}



\conferenceinfo{Under Submission}{Do Not Distribute}

\title{Jalangi: A Tool Framework for Concolic Testing, Selective
  Record-Replay, and Dynamic Analysis of JavaScript}
\numberofauthors{2}

\author{
\alignauthor
Koushik Sen\titlenote{The work of this author was supported in full by
  Samsung Research America.}\\
      \affaddr{EECS Department}\\
      \affaddr{UC Berkeley, CA, USA.}\\
     \affaddr{{\tt\small ksen@cs.berkeley.edu}}
\alignauthor
Swaroop Kalasapur, Tasneem Brutch,  and Simon Gibbs\\
\affaddr{Samsung Research America}\\
\affaddr{75 West Plumeria Drive, San Jose, CA, USA}\\
\affaddr{{\tt\small \{s.kalasapur,t.brutch,s.gibbs\}@sisa.samsung.com}}
}
% 2nd. author
\maketitle
\sloppy

\begin{abstract}
  We describe a tool framework, called \jalangi{}, for dynamic
  analysis and concolic testing of JavaScript programs.  The framework
  is written in JavaScript and allows implementation of various
  heavy-weight dynamic analyses for JavaScript. \jalangi{}
  incorporates two key techniques: 1) selective record-replay, a
  technique which enables to record and to faithfully replay a
  user-selected part of the program, and 2) shadow values and shadow
  execution, which enables easy implementation of heavy-weight dynamic
  analyses such as concolic testing and taint tracking.  \jalangi{}
  works through source-code instrumentation which makes it portable
  across platforms.  \jalangi{} is available at
  \url{https://github.com/SRA-SiliconValley/jalangi} under Apache 2.0
  license.  Our evaluation of \jalangi{} on the SunSpider benchmark
  suite and on five web applications shows that \jalangi{} has an
  average slowdown of 26X during recording and 30X slowdown during
  replay and analysis.  The slowdowns are comparable with slowdowns
  reported for similar tools, such as PIN and Valgrind for x86
  binaries.
\end{abstract}

\section{Introduction}

JavaScript is widely used for writing client-side web applications and
is getting increasingly popular for writing mobile applications.
However, unlike C, C++, and Java, there are not that many tools
available for analysis and testing of JavaScript applications.  We
have developed a simple yet powerful framework, called \jalangi{}, for
writing heavy-weight dynamic analyses for JavaScript.  In this paper,
we briefly describe the framework and its usage scenarios.  The
framework provides a few useful abstractions and an API that
significantly simplifies implementation of dynamic analyses for
JavaScript.  A detailed description of the techniques underlying
\jalangi{} can be found in~\cite{SBGKfse13}.

\jalangi{} works on any browser or \texttt{node.js}.  We achieve
browser independence through selective source
instrumentation. \emph{An attractive feature of \jalangi{} is that it
  can operate even if certain source files are not instrumented.}  An
analysis in \jalangi{} works in two-phases.  In the first phase, an
instrumented JavaScript application is executed and recorded on a user
selected platform (e.g. mobile chrome running on Android).  In the
second phase, the recorded data is utilized to perform a user
specified dynamic analysis in a desktop environment.

\jalangi{} allows easy implementation of a dynamic analysis through
the support of \emph{shadow values} and \emph{shadow execution}.
\emph{Shadow values} enable us to associate a shadow value with any
value used in the program.  A shadow value can contain useful
information about the actual value (e.g. taint information or symbolic
representation of the actual value).  The framework supports
\emph{shadow execution} on shadow values, a technique in which an
analysis can update the shadow values and analysis state, on each
operation performed by the actual execution.  For example, a shadow
execution can perform symbolic execution or dynamic taint propagation.

In \jalangi{}, we have implemented several dynamic analyses:

\begin{itemize}
\item Concolic Testing~\cite{dart,cute}: Concolic testing performs
  symbolic execution along a concrete execution path, generates a
  logical formula denoting a constraint on the input values, and
  solves a constraint to generate new test inputs that would execute
  the program along previously unexplored paths.  Our implementation
  of concolic testing supports constraints over integer, string, and
  object types and \emph{novel type constraints.}
\item Tracking origins of \texttt{null} and
  \texttt{undefined}~\cite{Bond:2007:TBA:1297027.1297057}: This
  analysis records source code locations where null and undefined
  values come into existence and reports them if they cause an error.
  Whenever there is an error due to such literals, such as accessing
  the field of a null value, the shadow value of the literal is
  reported to the user.  Such a report helps the programmer to easily
  identify the origin of the null value.
\item Detecting Likely Type Inconsistencies: The dynamic analysis
  checks if an object created at a given program location can assume
  multiple inconsistent type.
% computes the types of object and function values created at each
%   definition site in the program.  If an object or a function value
%   defined at a program location has been observed to assume more than
%   one type, then the analysis reports the program location along with
%   the observed types.  
  Sometimes these kind of type inconsistencies could point us to a
  potential bug in the program.  We have noticed such issues in two
  SunSpider benchmark programs.
\item Simple Object Allocation Profiler: This dynamic analysis
  computes the number of objects created at a given allocation site
  and how often the object has been accessed.% records
  % the number of objects created at an allocation site in the program.
  % It reports if the objects created at a given allocation site are
  % read-only or a constant.  It also reports the maximum and average
  % difference between the object creation time and the most recent
  % access time of the object
  .  If an allocation site creates too many contant objects, then it
  could lead to memory inefficiency.  We have found such a problem in
  one of the web applications in our benchmark suite.
\item Dynamic taint
  analysis~\cite{songndss05}: A
  dynamic taint analysis is a form of information flow analysis which
  checks if information can flow from a specific set of memory
  locations, called sources, to another set of memory locations,
  called sinks.  We have implemented a simple form of dynamic taint
  analysis in \jalangi{}.  In the analysis, we treat read of any field
  of any object, which has not previously been written by the
  instrumented source, as a source of taint.  We treat any read of a
  memory location that could change the control-flow of the program as
  a sink.  We attach taint information with the shadow value of an
  actual value.
\end{itemize}

\section{Technical Details}
\label{sec:technical-details}

In this section, we briefly summarize \jalangi{}'s underlying
techniques.  A detailed discussion of the techniques could be found
in~\cite{SBGKfse13}.  We assume that the user of \jalangi{} selects a
subset of the JavaScript source in a web application for
record-replay.  \jalangi{} instruments the user-selected source for
record-replay.  During the \emph{recording} phase, the application is
executed with the instrumented files on a platform of the user's
choice (e.g. a mobile browser or a node.js interpreter).  During
recording, the entire application is executed, i.e. all instrumented
and un-instrumented JavaScript files and native codes get executed.
During the \emph{replay} phase, \jalangi{} only replays the execution
of the instrumented sections.  This asymmetry of execution in the two
phases enables one to record an execution of a JavaScript application
on an actual platform (e.g. a mobile browser) and then replay the
execution for the purpose of debugging on a desktop JavaScript engine,
such as node.js or a JavaScript engine embedded in an IDE.  Moreover,
during replay, since we avoid the execution of un-instrumented code and
native code, we can easily implement various dynamic analyses that
depend on shadow values and shadow executions.

A trivial way to perform faithful record-replay of an execution is to
record every value loaded from memory during an execution and use
those values for corresponding memory loads in the replay phase.  This
approach has two challenges: 1) How do we record values of objects and
functions?  2) How do we replay an execution when an un-instrumented
function or a native function, such as the JavaScript event
dispatcher, calls an instrumented function?  Note that we do not allow
the execution of un-instrumented and native functions during the
replay phase.  Therefore, we need an alternative mechanism to execute
instrumented functions that are being invoked by un-instrumented
functions during recording.  We address the first challenge by
associating a unique numerical identifier with every object and
function and by recording the value of those unique identifiers.  We
address the second challenge by explicitly recording and calling
instrumented functions that are being invoked from un-instrumented
functions or are dispatched by the JavaScript event dispatcher.


In \jalangi{}, we avoid recording of every load of memory.  \emph{This
  is based on the observation that if we can compute the value of a
  memory load during the replay phase by solely executing the
  instrumented code, then we do not need to record the value of the
  load.}  In order to determine if the value of a memory load needs to
be recorded, \jalangi{} maintains a shadow memory during the recording
phase.  The shadow memory is updated along with the actual memory
during the execution of instrumented code.  Execution of
un-instrumented and native code does not update the shadow memory.
During the load of a memory location in the recording phase, if
\jalangi{} finds any difference between the value of the actual memory
being loaded and the value stored in the corresponding shadow memory,
\jalangi{} records the value of such memory loads. This ensures that
correct values are available during the replay phase.

In shadow execution, \jalangi{} allows the replacement of any value
used in the execution by an \emph{annotated value}.  The annotated
value can carry extra information about the actual value.  For
example, an annotated value can carry taint information in a taint
analysis or a symbolic expression describing the actual value in
symbolic execution.  In \jalangi{}, we denote an annotated value using
an object of type \texttt{ConcolicValue}.  An object of type
\texttt{ConcolicValue} has two fields: the field \texttt{concrete}
stores the actual value and the field \texttt{symbolic} stores the
shadow value, i.e. extra information about the actual value.  A value,
say $v$, in JavaScript can be associated with shadow value, say $s$,
by simply replacing $v$ by \texttt{new ConcolicValue}$(v, s)$.  The
projection function $\Actual(v)$ returns the actual value of $v$, if
$v$ is an annotated value and returns $v$ otherwise.  Similarly, the
projection function $\Shadow(v)$ returns the shadow value associated
with $v$ if $v$ is an annotated value and returns \texttt{undefined}
otherwise.


\section{Implementation}
\label{sec:implementation}

\jalangi{} is available at
\url{https://github.com/SRA-SiliconValley/jalangi}.  We have
implemented \jalangi{} in JavaScript.

\begin{table*}
\begin{minipage}{0.28\textwidth}
\begin{tabular}{|r|}
\hline\\
{\scriptsize
\begin{lstlisting}
var a = {x:1, y:2};

function f2 (c) {
 if (c > 5 ) 
   a.y = a.x + c;
 return c;
}

f1(12);
\end{lstlisting}
}\\
\hline
\end{tabular}
\caption{Sample Code Before Instrumentation}
\label{tab:original}
\end{minipage}
\begin{minipage}{0.71\textwidth}
\begin{tabular}{|r|}
\hline\\
{\scriptsize
\begin{lstlisting}
  J$.U(iid, op, oprnd); // wrapper for unary operations
  J$.B(iid, op, left, right); // wrapper for binary operations
  J$.C(iid, cond); // wrapper for conditional branches
  J$.C1(iid, key); // wrapper for the key of a switch statement
  J$.C2(iid, case); // wrapper for a case label of a switch
  J$._();  // returns last value passed to J$.C

  J$.H(iid, val); // wrapper for hash used in for-in
  J$.I(val); // ignore argument
  J$.G(iid, base, offset); // wrapper for getField
  J$.P(iid, base, offset, val); // wrapper for putField
  J$.R(iid, name, val); // wrapper for local variable read
  J$.W(iid, name, val, lhs); // wrapper for local variable write
  J$.Niid, name, val, isArgument); // wrapper for initialization
  J$.T(iid, val, type); // wrapper for a object/function/regexp/array literal
  J$.F(iid, f, isConstructor); // wrapper for a function call
  J$.M(iid, base, offset, isConstructor); // wrapper for a method call
  J$.A(iid, base, offset, op); // wrapper for +=, -=, ...
  J$.Fe(iid, val, dis); // callback at function entry
  J$.Fr(iid); // callback at function return
  J$.Se(iid, val); // callback at script entry
  J$.Sr(iid); // callback at script exit
  J$.Rt(iid, val); // wrapper for value being returned
  J$.Ra();  // callback for grabbing return value

  J$.makeSymbolic(symbol, val);  // make a value symbolic
  J$.addAxiom(formula, branch);  // adds a constraint to the path constraint
  J$.endExecution(); // callback at the end of an execution

\end{lstlisting}
}\\
\hline
\end{tabular}
\caption{Callback Functions from Instrumented Code}
\label{tab:callbacks}
\end{minipage}
\end{table*}

\begin{table*}
\begin{tabular}{|r|}
\hline\\
{\scriptsize
\begin{lstlisting}
if (typeof window === 'undefined') {
  require('/user/jalangi/src/js/analysis.js');
  require('/user/jalangi/src/js/InputManager.js');
  require('/user/jalangi/src/js/instrument/esnstrument.js');
  require(process.cwd() + '/inputs.js');
}
{
  try {
    J$.Se(73, 'tests/unit/instrument-small_jalangi_.js');
    J$.N(77, 'a', a, false);
    J$.N(85, 'f2', J$.T(81, f2, 12), false);
    var a = J$.W(17, 'a', J$.T(13, {
      x: J$.T(5, 1, 22),
      y: J$.T(9, 2, 22)
    }, 11), a);
    function f2(c) {
      jalangiLabel0:
        while (true) {
          try {
            J$.Fe(49, arguments.callee, this);
            J$.N(53, 'arguments', arguments, true);
            J$.N(57, 'c', c, true);
            if (J$.C(4, J$.B(6, '>', J$.R(21, 'c', c), J$.T(25, 5, 22))))
              J$.P(45,J$.R(29,'a',a),'y',J$.B(10,'+',J$.G(37,J$.R(33,'a',a),'x'),J$.R(41,'c',c)));
            return J$.Rt(97, J$.R(49, 'c', c));
          } catch (J$e) {
            throw J$e;
          } finally {
            if (J$.Fr(93))
              continue jalangiLabel0;
            else
              return J$.Ra();
          }
        }
    }
    J$.F(69,J$.I(typeof f1==='undefined'?J$.R(61,'f1',undefined):J$.R(61,'f1',f1)),false)(J$.T(65,12,22));
  } catch (J$e) {
    throw J$e;
  } finally {
    J$.Sr(89);
  }
}
// JALANGI DO NOT INSTRUMENT

//@ sourceMappingURL=instrument-small_jalangi_.js.map
\end{lstlisting}
}\\
\hline
\end{tabular}
\caption{After Instrumentation of  Code in Table~\ref{tab:original}}
\label{tab:instrumented}
\end{table*}

\jalangi{} operates by instrumenting JavaScript code.
Table~\ref{tab:instrumented} shows code obtained after instrumentation
of the code in Table~\ref{tab:original}.  During instrumentation,
\jalangi{} inserts various callback functions from the \jalangi{}
library.  The callback functions are listed in
Table~\ref{tab:callbacks}.  These functions wrap the various
operations in JavsScript.  The selective record-replay engine of
\jalangi{} is implemented by defining these callback functions.

\jalangi{} exposes the instrumentation library as a function
\texttt{instrumentCode}.  This enables us also to dynamically
instrument any code that is created and evaluated at runtime.  For
example, we modify any call to \texttt{eval(s)} to
\texttt{eval(instrumentCode(s))}.

During recording phase, \jalangi{} generates a \texttt{trace} array
which contains all recorded values needed for replay.  \jalangi{}
serializes the \texttt{trace} array in JSON format.  \jalangi{} stores
the serialized array in a file, or sends it to a server for storage.
Replay uses the serialized array stored in the file to initialize the
\texttt{trace} array. Replay can be performed through an IDE or a
stand-alone application (realized with node.js for example). This
enables us to perform heavy-weight debugging or analysis of a recorded
execution outside a browser.

\section{A Quick Guide to Jalangi Installation and Usage}
\label{sec:usage}

\jalangi{} is a command-line tool.  We also have a prototype Eclipse
plugin to invoke the various \jalangi{} commands from Eclipse.  A
screencast of \jalangi{} running as an Eclipse plugin is available at
\url{http://srl.cs.berkeley.edu/~ksen/jalangi.html}.  An extended
version of the demo in the screencast will be used for demonstration
at the conference.

\subsection{Platform Requirements}
\label{sec:platf-requ}

We have tested \jalangi{} on Mac OS X 10.8. \jalangi{} should work on
Mac OS 10.7 and higher and Ubuntu 11.0 and higher.  \jalangi{} has
dependencies on the following:

\begin{itemize}
\item Latest version of Node.js available at
  \url{http://nodejs.org/}. We have tested \jalangi{} with Node v0.8.22
  and v0.10.4.
\item Sun's JDK 1.6 or higher. We have tested \jalangi{} with Java
  1.6.0\_43.
\item Command-line git.  We use git to download various dependencies.
\item libgmp (\url{http://gmplib.org/}) is required by
  cvc3~\cite{BT07}. Concolic testing uses cvc3 and automaton.jar
  (\url{http://www.brics.dk/automaton/}) for constraint solving. The
  installation script checks if cvc3 and automaton.jar are installed
  properly.
\item Chrome browser if you need to test web apps.
\end{itemize}

\subsection{Installation}
\label{sec:installation}

\jalangi{} can be downloaded and installed as follows.

\begin{verbatim}
git clone https://github.com/SRA-SiliconValley/jalangi
cd jalangi
./scripts/install
./scripts/testsym
\end{verbatim}

\subsection{Usage}
\label{sec:usage-1}

\subsubsection*{Concolic Testing}
\label{sec:concolic-testing}

To perform concolic testing of some JavaScript code present in a file,
say testme.js, one need to insert the following four lines at the top
of the file.

\begin{verbatim}
if (typeof window === "undefined") {
    require('../../src/js/InputManager');
    require(process.cwd()+'/inputs');
}
\end{verbatim}

In the code, an input can be specified by calling \texttt{
  J\$.readInput(arg)}. Concolic testing can be performed by calling
the script \texttt{./scripts/concolic} as follows.

\begin{verbatim}
./scripts/concolic testme 100000
\end{verbatim}

The third argument bounds the total number of test inputs generated by
concolic testing. The command generates a set of input files in the
directory \texttt{jalangi\_tmp}. The input files start with the prefix
\texttt{jalangi\_inputs}.  Once the inputs are generated,
\texttt{testme.js} can be run on those inputs by giving the following
command:

\begin{verbatim}
./scripts/rerunall testme
\end{verbatim}

For example, one can generate inputs for the code in the file
tests/unit/regex8.js by running the following commands.

\begin{verbatim}
./scripts/concolic tests/unit/regex8 100
./scripts/rerunall tests/unit/regex8
\end{verbatim}

\subsubsection*{Dynamic Analysis}
\label{sec:dynamic-analysis}

\jalangi{} provides a number of builtin dynamic analyses.  The
analyses can be found under the directory \texttt{src/js/analysis/}.
A template for an analysis is provided in the file
\texttt{src/js/analysis/nop/NOPEngine.js}.  A new analysis can be
implemented from the template by defining the various callback
functions in the template.  A simple analysis can be found in the file
\texttt{src/js/simpletaint/SimpleTaintEngine.js}.  An analysis file
name must end with \texttt{Engine.js}.  The JavaScript code in \texttt{ObjectAllocationTrackerEngine.js} under \texttt{src/js/analyses/objectalloc/}
implements the object allocation profiler. The analysis can be
performed on a file, say \texttt{tests/sunspider1/crypto-aes.js}, by
invoking the following command:

\begin{verbatim}
./scripts/analysis analyses/objectalloc/Object
AllocationTrackerEngine tests/sunspider1/crypto-aes
\end{verbatim}

Similarly, the likely type inference analysis on another sunspider
benchmark can be performed by calling the following command.

\begin{verbatim}
./scripts/analysis analyses/likelytype/LikelyType
InferEngine tests/sunspider1/crypto-sha1
\end{verbatim}


\section{Conclusion}
\label{sec:conclusion}

\jalangi{} has taken care of various challenging details of JavaScript.  One
can easily implement a dynamic analysis in the \jalangi{} framework
without worrying about the worrysome corners of JavaScript.  We expect that
\jalangi{} will facilitate future research on dynamic analysis of
JavaScript. 

% \subsubsection*{Record and replay a web application}
% \label{sec:record-replay-web}

% First start a HTTP server by running the following command. The command starts a simple Python based http server.

% \begin{verbatim}
% python -m SimpleHTTPServer \&
% \end{verbatim}
% Then one needs to manually instrument the JavaScript files that
% require analysis. The \texttt{index.html} also needs to edited so that
% it loads the \jalangi{} library files and the instrumented files.

% \begin{verbatim}
% node src/js/instrument/esnstrument.js tests/tizen/annex/js/annex.js
% \end{verbatim}
% Finally launch the jalangi server and the html page by running

% \begin{verbatim}
% ./scripts/rrserver http://127.0.0.1:8000/tests/tizen/annex/index_jalangi_.html
% \end{verbatim}

% You can now play the game for sometime. Try two moves. This will
% generate a jalangi\_trace1 in the current directory. You can run a
% dynamic analysis on the trace file by issuing the following commands.

{\small
\bibliographystyle{abbrv}
\bibliography{jalangi}
}
\end{document}

